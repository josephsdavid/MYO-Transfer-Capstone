\section{Conclusions}

In this study, a novel method for classifying sEMG signals for myoelectric control was highlighted. It was shown that using a simple attention mechanism and state of the art training techniques, a model can be built to accurately classify a broad range of gestures, in a short period of time. This has immediate applications in the field of prosthetics, and can be expanded to other body parts with relatively little difficulty. \par
One general flaw of this study in general is the use of supervised learning at all. The presence of labels in human motion is rather unintuitive. In the future, research should be expanded into reinforcement learning and imitation learning, potentially combined with computer vision, in order to develop more robust models with an infinite range of gestures.

