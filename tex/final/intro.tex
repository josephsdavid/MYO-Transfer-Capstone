\section{Introduction}

Electrophysiological studies of the nervous system are the core area of research in clinical neurophysiology, where scientists attempt to link electrical signals from the body to real world effects. These studies include measuring brain waves (electroencephalography), comparison of sensory stimuli to electrical signals in the central nervous system (evoked potential), and the measure of electrical signals in skeletal muscles (electromyography). \par
Electromyography is of particular interest to this paper. The nervous system uses electrical signals to communicate with the rest of the body. When a signal from the nervous system reaches a skeletal muscle, the myocites (muscle cells) contract, causing a physical motion. By measuring these electrical signals in a supervised manner, we can develop a link between signal and physical action. This connection yields many powerful uses, ranging from quantifying physical veracity to diagnosing neurodegenerative diseases. An example of the latter can be found in Akhmadeev et al. \cite{graves}, where electromyographic (EMG) signals were used to classify Multiple Sclerosis patients from healthy control subjects with 82\% accuracy. \par
Deep learning can be used to further improve the power and utility of the EMG analysis. A deep neural network is, in essence, a composition of neurons (regressors) that learns a functional mapping between two sets of data. By learning a mapping between EMG signal and physical effect, we can develop more sensitive and accurate models of what connects the two. This also allows us to use less intrusive measurement devices in studies and in the real world. The applications of this range from clinical trials and prognostication of neuromuscular diseases to gesture prediction in "brain-controlled" prosthetic limbs. This study focuses on the latter application.
Current state of the art myoelectric prosthetic limbs  are capable of detecting and performing between 7 and 18 motions \cite{myohandpro} \cite{ottoblock}. While this represents a significant increase in the quality of life of an amputee, there remains a large amount of room for improvement. Most of the potential for improvement does not lie in the robotics themselves, which are fairly robust, but within the device which maps brain signal to motion of the hand. The aim of this paper is therefore to build a highly accurate gesture classifier using EMG signal, capable of classifying a broad range of gestures. \par
 Extending from this initial goal, there are certain metrics which must be met for the classifier to be deemed "useful". First, it must be fast. If there is a large degree of latency from thought to hand motion, the user will simply not use the device. Thus, it is important to determine the time window in which a prediction must be made. The absolute largest prediction latency for a model to be considered useful for myoelectric control lies between 250 and 300 milliseconds  \cite{300ms}, \cite{250ms}. For this study, the precedent set by \cite{primary} was followed, with a 260 millisecond prediction window.
Another key consideration for the classifier is generalizability. Within the field of myoelectric control and gesture recognition, there are two important types of generalizability, intra-subject and inter-subject \cite{ring_2018}. Intra subject generalizability indicates that the model is robust to personal elements, such as muscle fatigue. In contrast, inter-subject generalizability refers to the ability of the model to generalize to new people. This study focuses on intra-subject generalizability, as it is important that the device keep working for extended periods of time, while models can be fit to and personalized for amputees and nonamputees alike \cite{amputeedb}. \par
In this paper, we propose a novel attentional architecture for sEMG recognition and myoelectric control. We demonstrate the model's validity on the 53-class NinaPro DB5 \cite{nina5}. We also compare different techniques  for dealing with the inherent class imbalance in sEMG, a synthetic data based approach (augmentation), an undersampling based approach, and a loss-based approach. All methods are evaluated on an intra-subject basis using the same holdout samples, yielding promising results.






%The absolute largest prediction latency for a model to be considered useful for myoelectric control lies between 250 and 300 milliseconds  \cite{300ms}, \cite{250ms}. Many recent papers claim excellent gesture prediction results, however they require time samples (windows) between 0.5 and 1.5 seconds long \cite{rnn_1000}, \cite{rnn_128}. In this paper, all windows are 260 milliseconds in length, following the precedent set by Allard et al. \cite{primary}. This leaves 40 milliseconds for any transformations done on the window, as well as the amount of time it takes for the model to make a prediction.


